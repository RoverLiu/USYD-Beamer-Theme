%--------------------------------------------------------------------
%                           Introduction
%--------------------------------------------------------------------
% This is an example of an unofficial University of Sydney LaTeX Beamer
% template & theme for the School of Mathematics and Statistics.

% Created by: Daniel Daners & Samantha Clarke
% Version: 0.1
% Updated: 31 May 2017
% Complied with: pdflatex

% GitHub repo: https://github.com/samanthalclarke/USYD-Beamer-Theme

% Please see GitHub repo for current version, README.md, & licence.md files. 

% This file can be redistributed and/or modified under the terms of the GNU
% Public License, version 3.

% Please note that this template is still under development. Any comments,
% feedback, additions, or suggestions are welcome. 

%--------------------------------------------------------------------
%                           Licence
%--------------------------------------------------------------------
% Copyright (c) 2017 Daniel Daners & Samantha Clarke

% Permission is hereby granted, free of charge, to any person obtaining a copy
% of this software and associated documentation files (the "Software"), to deal
% in the Software without restriction, including without limitation the rights
% to use, copy, modify, merge, publish, distribute, sublicense, and/or sell
% copies of the Software, and to permit persons to whom the Software is
% furnished to do so, subject to the following conditions:

% The above copyright notice and this permission notice shall be included in all
% copies or substantial portions of the Software.

% THE SOFTWARE IS PROVIDED "AS IS", WITHOUT WARRANTY OF ANY KIND, EXPRESS OR
% IMPLIED, INCLUDING BUT NOT LIMITED TO THE WARRANTIES OF MERCHANTABILITY,
% FITNESS FOR A PARTICULAR PURPOSE AND NONINFRINGEMENT. IN NO EVENT SHALL THE
% AUTHORS OR COPYRIGHT HOLDERS BE LIABLE FOR ANY CLAIM, DAMAGES OR OTHER
% LIABILITY, WHETHER IN AN ACTION OF CONTRACT, TORT OR OTHERWISE, ARISING FROM,
% OUT OF OR IN CONNECTION WITH THE SOFTWARE OR THE USE OR OTHER DEALINGS IN THE
% SOFTWARE.

%--------------------------------------------------------------------
%                           Document Setup
%--------------------------------------------------------------------

%\documentclass{beamer}
\documentclass[handout]{beamer} 	%to produce printable hand-outs
% user packages
\usepackage{amsmath,amssymb}

%load theme
\usetheme{usyd2016}

% use if you want greyed out steps rather than invisible
%\setbeamercovered{transparent}	

\usetheme{usyd2016}

%------------------------------ Title page setup ------------------------------%

% document details
\title{Presentation Title}
\subtitle{Presentation Subtitle} % (optional)
\author{Professor Firstname Lastname}
\institute{Faculty, Centre, or Unit}
\date{\today} % (optional)


%--------------------------------------------------------------------
%                           Begin Document
%--------------------------------------------------------------------

\begin{document}

\begin{frame}
  \titlepage
\end{frame}

%------------------------------ Slide ------------------------------%

\begin{frame}{Slide Heading A}

Here is a list:
\begin{itemize}
\item Item X.
\item Item Y.
\end{itemize}

\bigskip

A numbered list:
\begin{enumerate}
\item Point 1
\item Point 2
\end{enumerate}

\end{frame}

%-------------------------------------------------------------------
%                          Section 1
%-------------------------------------------------------------------

%--------------------------------------------------------------------
%                           Section Divider Slide 1
%--------------------------------------------------------------------

\section{Section Divider Heading 1}
\begin{frame}
  \sectionpage
\end{frame}
%------------------------------ Slide ------------------------------%

\begin{frame} 
\frametitle{Slide Heading B} 
\framesubtitle{The proof uses \textit{reductio ad absurdum}.} 

\begin{theorem}
There is no largest prime number. \end{theorem} 
\begin{enumerate} 
\item<1-| alert@1> Suppose $p$ were the largest prime number. 
\item<2-> Let $q$ be the product of the first $p$ numbers. 
\item<3-> Then $q+1$ is not divisible by any of them. 
\item<1-> But $q + 1$ is greater than $1$, thus divisible by some prime
number not in the first $p$ numbers.
\end{enumerate}

\end{frame}

%-------------------------------------------------------------------
%                          Section 2
%-------------------------------------------------------------------

%--------------------------------------------------------------------
%                           Section Divider Slide 2
%--------------------------------------------------------------------

\section{Section Divider Heading 1}
\begin{frame}
  \sectionpage
\end{frame}

%------------------------------ Slide ------------------------------%

\begin{frame}
\frametitle{Sample frame title}
This is a text in first frame. This is a text in first frame. This is a text in first frame.
\end{frame}

%------------------------------ Slide ------------------------------%

\begin{frame}{Make Titles Informative.}

  You can create overlays\dots
  \begin{itemize}
  \item using the \texttt{pause} command:
    \begin{itemize}
    \item
      First item.
      \pause
    \item    
      Second item.
    \end{itemize}
  \item
    using overlay specifications:
    \begin{itemize}
    \item<3->
      First item.
    \item<4->
      Second item.
    \end{itemize}
  \item
    using the general \texttt{uncover} command:
    \begin{itemize}
      \uncover<5->{\item
        First item.}
      \uncover<6->{\item
        Second item.}
    \end{itemize}
  \end{itemize}
\end{frame}

%-------------------------------------------------------------------
%                          Section Summary
%-------------------------------------------------------------------

% The background settings from Section 2 will continue to apply here.

\section*{Summary}

%------------------------------ Slide ------------------------------%

\begin{frame}{Summary}

  % Keep the summary *very short*.
  \begin{itemize}
  \item
    The \alert{first main message} of your talk in one or two lines.
  \item
    The \alert{second main message} of your talk in one or two lines.
  \item
    Perhaps a \alert{third message}, but not more than that.
  \end{itemize}
  
  % The following outlook is optional.
  \vskip0pt plus.5fill
  \begin{itemize}
  \item
    Outlook
    \begin{itemize}
    \item
      Something you haven't solved.
    \item
      Something else you haven't solved.
    \end{itemize}
  \end{itemize}
\end{frame}


%--------------------------------------------------------------------
%                           End Document
%--------------------------------------------------------------------

\end{document}


%%% Local Variables:
%%% mode: latex
%%% TeX-master: t
%%% End:
